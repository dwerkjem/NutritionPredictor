\documentclass{article}

\title{Notes On Data}
\author{Derek R Neilson}
\date{\today}

\usepackage{graphicx} % For including images
\usepackage{amsmath}  % For mathematical expressions
\usepackage{hyperref} % For hyperlinks
\usepackage{booktabs} % For better tables
\usepackage{listings} % For including code snippets
\usepackage{url}      % For URLs
\usepackage{xcolor} % For color definitions

% Set global listings settings (optional)
\lstset{
    language=bash,
    basicstyle=\ttfamily\small, % Use a monospaced font
    xleftmargin=0pt,            % Remove left margin
    frame=single,               % Add a frame around the code
    breaklines=true,            % Allow line breaking
    postbreak=\mbox{\textcolor{red}{$\hookrightarrow$}\space}, % Indicate line breaks
}

\begin{document}

\maketitle

\begin{abstract}
  This document contains notes on the data. The notes are intended to demonstrate how I filter and manipulate the data, and are purely for my instructor to review.
\end{abstract}

\section{Introduction}

In data analysis, the ability to effectively filter and manipulate data is crucial for extracting meaningful insights. This document outlines the methodologies and tools I employ to pre-process and analyze the dataset. The primary focus is on cleaning the data, handling missing values, and transforming data to suit the analytical objectives. These notes serve as a comprehensive guide for understanding my data processing workflow.

\section{Data Collection}
The data was collected from \url{https://fdc.nal.usda.gov/}. The dataset is 2.9GB and is labeled \texttt{Branded} and is in JSON format. I chose this dataset because it is large and one can assume that it has the most rows because it is so large. 

To download the data, I used the following commands: 

\begin{lstlisting}
wget https://fdc.nal.usda.gov/fdc-datasets/FoodData_Central_branded_food_csv_2024-04-18.zip
unzip FoodData_Central_branded_food_csv_2024-04-18.zip 
\end{lstlisting}



\end{document}

