\documentclass{article}

\title{Notes On Data}
\author{Derek R Neilson}
\date{\today}

\usepackage{graphicx} % For including images
\usepackage{amsmath}  % For mathematical expressions
\usepackage{hyperref} % For hyperlinks
\usepackage{booktabs} % For better tables
\usepackage{url}      % For URLs
\usepackage{listings}   % For code listings
\usepackage{xcolor}     % For custom colors (optional)

\lstset{
    basicstyle=\ttfamily\small,
    keywordstyle=\color{blue},
    commentstyle=\color{gray},
    stringstyle=\color{orange},
    numbers=left,
    numberstyle=\tiny\color{gray},
    stepnumber=1,
    numbersep=5pt,
    backgroundcolor=\color{white},
    frame=single,
    breaklines=true,
    captionpos=b,        % Position of the caption (b=below, t=top)
    language=bash        % Specify the language for syntax highlighting

\begin{document}

\maketitle

\begin{abstract}
  This document contains notes on the data. The notes are intended to demonstrate how I filter and manipulate the data, and are purely for my instructor to review.
\end{abstract}

\section{Introduction}

In data analysis, the ability to effectively filter and manipulate data is crucial for extracting meaningful insights. This document outlines the methodologies and tools I employ to pre-process and analyze the dataset. The primary focus is on cleaning the data, handling missing values, and transforming data to suit the analytical objectives. These notes serve as a comprehensive guide for understanding my data processing workflow.

\section{Data Collection}
The data was collected from \url{https://fdc.nal.usda.gov/}. The dataset is 2.9GB and is labeled \texttt{Branded} and is in JSON format. I chose this dataset because it is large and one can assume that it has the most rows because it is so large. 

To download the data, I used the following commands: 


\
baegin{lstlisting}[caption={Download, Extract, and Remove Dataset File}, label={lst:download_extract}]
wget https://fdc.nal.usda.gov/fdc-datasets/FoodData_Central_branded_food_json_2024-04-18.zip
unzip FoodData_Central_branded_food_json_2024-04-18.zip 
rm FoodData_Central_branded_food_json_2024-04-18.zip
\end{lstlisting}

As shown in Listing~\ref{lst:download_extract}, the commands download, extract, and remove the dataset file.

It is worth noteing that I am using git to track changes in the code and data. The git commands will not be shown in this documen for brevity sake.

\section{Data Inspection}
I receved the following files after extracting:
  \begin{itemize}
    \item \texttt{brandedDownload.json} I am assuming that this is the main file
    \item \texttt{foundationDownload.json} I am assuming that this is a supporting file
  \end{itemize}


\end{document}

