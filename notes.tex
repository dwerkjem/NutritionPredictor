
\documentclass{report} % Changed from 'article' to 'report'

\title{Notes On Data}
\author{Derek R Neilson}
\date{\today}

\usepackage{geometry}    % Adjust margins
\usepackage{graphicx}    % For including images
\usepackage{amsmath}     % For mathematical expressions
\usepackage{hyperref}    % For hyperlinks
\usepackage{booktabs}    % For better tables
\usepackage{url}         % For URLs
\usepackage{listings}    % For code listings
\usepackage{xcolor}      % For custom colors (optional)
\usepackage{hyphenat}    % For hyphenation

% Adjust margins
\geometry{
    a4paper,
    margin=1in
}

% Base style for all listings
\lstdefinestyle{baseStyle}{
    basicstyle=\ttfamily\small,          % Use a smaller monospace font
    numbers=left,                        % Line numbers on the left
    numberstyle=\tiny\color{gray},       % Style for line numbers
    stepnumber=1,                        % Number every line
    numbersep=5pt,                       % Space between numbers and code
    backgroundcolor=\color{white},       % Background color
    frame=single,                        % Single frame around code
    breaklines=true,                     % Allow line breaking
    breakatwhitespace=false,            % Allow breaks at any character
    breakautoindent=true,
    breakindent=10pt,                    % Indent for broken lines
    columns=flexible,                    % Better alignment
    keepspaces=true,                     % Preserve spaces
    showstringspaces=false,              % Don't show spaces in strings
    tabsize=4,                           % Set tab size
    captionpos=b,                        % Caption below the listing
    postbreak=\mbox{\textcolor{red}{$\hookrightarrow$}\space}, % Symbol indicating a line break
    literate={/}{/}1{.}{.}1{-}{-}1{_}{\_}1,  % Allow breaks after / . - _
}

% Style for Python
\lstdefinestyle{pythonStyle}{
    style=baseStyle,                     % Inherit from baseStyle
    language=Python,                     % Set language
    keywordstyle=\color{blue}\bfseries,  % Keywords in blue bold
    commentstyle=\color{gray}\itshape,   % Comments in gray italic
    stringstyle=\color{orange},          % Strings in orange
    morekeywords={self, cls, async, await}, % Additional Python keywords
}

% Style for R
\lstdefinestyle{rStyle}{
    style=baseStyle,
    language=R,
    keywordstyle=\color{purple}\bfseries,
    commentstyle=\color{gray}\itshape,
    stringstyle=\color{red},
    morekeywords={<-},                   % Assignment operator in R
}

% Style for Bash
\lstdefinestyle{bashStyle}{
    style=baseStyle,
    language=bash,
    keywordstyle=\color{blue}\bfseries,
    commentstyle=\color{gray}\itshape,
    stringstyle=\color{orange},
    morekeywords={wget, unzip, rm, sftp, ssh, put, bye}, % Additional Bash keywords
}

% Style for SQL (as an example of another data science-related language)
\lstdefinestyle{sqlStyle}{
    style=baseStyle,
    language=SQL,
    keywordstyle=\color{blue}\bfseries,
    commentstyle=\color{gray}\itshape,
    stringstyle=\color{orange},
    morekeywords={SELECT, FROM, WHERE, INSERT, UPDATE, DELETE, JOIN, INNER, OUTER},
}

% Style for Julia
\lstdefinestyle{juliaStyle}{
    style=baseStyle,
    language=Julia,
    keywordstyle=\color{teal}\bfseries,
    commentstyle=\color{gray}\itshape,
    stringstyle=\color{magenta},
    morekeywords={end, function, using, import},
}

% Style for MATLAB
\lstdefinestyle{matlabStyle}{
    style=baseStyle,
    language=Matlab,
    keywordstyle=\color{blue}\bfseries,
    commentstyle=\color{gray}\itshape,
    stringstyle=\color{orange},
    morekeywords={function, end, if, else, for, while},
}

% Style for LaTeX code
\lstdefinestyle{latexStyle}{
    style=baseStyle,
    language=TeX,
    keywordstyle=\color{blue}\bfseries,
    commentstyle=\color{gray}\itshape,
    stringstyle=\color{orange},
    morekeywords={\documentclass, \usepackage, \begin, \end, \section, \subsection},
}


\begin{document}

\maketitle

\begin{abstract}
  This document contains notes on the data. The notes are intended to demonstrate how I filter and manipulate the data. The primary focus is on cleaning the data, handling missing values, and transforming data to suit the analytical objectives. These notes serve as a comprehensive guide for understanding my data processing workflow. 
\end{abstract}

\chapter{Data Collection and Inspection}
\section{Data Collection} 
\subsection{Introduction}

In data analysis, the ability to effectively filter and manipulate data is crucial for extracting meaningful insights. This document outlines the methodologies and tools I employ to pre-process and analyze the dataset. The primary focus is on cleaning the data, handling missing values, and transforming data to suit the analytical objectives. These notes serve as a comprehensive guide for understanding my data processing workflow.

\subsection{Data Collection}

The data was collected from \url{https://fdc.nal.usda.gov/}. The dataset is 2.9GB and is labeled \texttt{Branded} and is in JSON format. I chose this dataset because it is large and one can assume that it has the most rows because it is so large. 

To download the data, I used the following commands: 

\begin{lstlisting}[style=bashStyle, caption={Download, Extract, and Remove Zip File}, label={lst:download_extract}]
wget https://fdc.nal.usda.gov/fdc-datasets/FoodData_Central_branded_food_json_2024-04-18.zip
unzip FoodData_Central_branded_food_json_2024-04-18.zip 
rm FoodData_Central_branded_food_json_2024-04-18.zip
\end{lstlisting}

As shown in Listing~\ref{lst:download_extract}, the commands download, extract, and remove the dataset file.

It is worth noting that I am using git to track changes in the code and data. The git commands will not be shown in this document for brevity.

\section{Data Inspection}
I received the following files after extracting:
  \begin{itemize}
    \item \texttt{brandedDownload.json} - I am assuming that this is the main file
    \item \texttt{foundationDownload.json} - I am assuming that this is a supporting file
  \end{itemize}

The first step to inspecting the data is to view it.

\begin{lstlisting}[style=bashStyle, caption={View the Data}, label={lst:view_data}]
less brandedDownload.json # the output is too large to show here and is not useful
# I am going to use jq to view the data
jq . brandedDownload.json # this results in a segmentation fault because the file is too large
# I am going to use a stronger server to view the data 
# For security reasons, the IP address and username are redacted
sftp -P port username@ip_address
put brandedDownload.json DEV/Project/Data
put foundationDownload.json DEV/Project/Data
bye
ssh username@ip_address -P port
\end{lstlisting}

As shown in Listing~\ref{lst:view_data}, the file is too large to view on my local machine. I will use a stronger server to view the data. Note that there is an assumption that all commands that follow are run on the server. From here on, I will refer to the server as being the machine that I am using to view the data. To get the data on the server, I used `sftp` to transfer the files to the server.

\begin{lstlisting}[style=bashStyle, caption={View the Data on the Server}, label={lst:view_data_server}]
jq . brandedDownload.json | less # failed because the file is too large
jq --stream . brandedDownload.json | less # this works because it streams the data
jq --stream . foundationDownload.json | less

\end{lstlisting}

After looking at the head of the data, I can see that the data is in unstructured JSON format. I will use the CSV data instead. The JSON data will not be included in this document for size reasons.

\section{Data Collection (CSV)}
\begin{lstlisting}[style=bashStyle, caption={Download, Extract, and Remove Zip File (CSV)}, label={lst:download_extract_csv}]
rm brandedDownload.json foundationDownload.json # remove the JSON files
wget https://fdc.nal.usda.gov/fdc-datasets/FoodData_Central_branded_food_csv_2024-04-18.zip # download the CSV file
unzip FoodData_Central_branded_food_csv_2024-04-18.zip # unzip the file
mv FoodData_Central_branded_food_csv_2024-04-18/* . # move the files to the current directory
\end{lstlisting}
As shown in Listing~\ref{lst:download_extract_csv}, the commands download, extract, and remove the dataset file in CSV format. I will use the CSV data for the rest of the analysis.

\begin{lstlisting}[style=pythonStyle, caption={Validate the Data}, label={lst:validate_data}]
import requests
import dotenv
import pandas as pd
import os
from scipy import stats
import numpy as np

# Load environment variables
dotenv.load_dotenv()

# Path to CSV
csv_file_path = "branded_food.csv"

# Check file existence
if not os.path.isfile(csv_file_path):
    raise FileNotFoundError(f"The file {csv_file_path} does not exist.")

# Load data
try:
    data = pd.read_csv(csv_file_path)
    print("Data loaded successfully.")
except Exception as e:
    print(f"An error occurred while loading the data: {e}")
    raise

print(f"Data shape: {data.shape}")

# 1. Inspect Data Types
print("\n--- Data Types ---")
print(data.dtypes)

# Convert specific columns if necessary
# Example: data['price'] = pd.to_numeric(data['price'], errors='coerce')

# 2. Check for Missing Values
print("\n--- Missing Values ---")
missing_values = data.isnull().sum()
print(missing_values)
missing_percentage = (missing_values / len(data)) * 100
print("\n--- Percentage of Missing Values ---")
print(missing_percentage)

# Handle missing values (Example: Fill numerical with mean, categorical with mode)
numerical_cols = data.select_dtypes(include=[np.number]).columns
categorical_cols = data.select_dtypes(include=["object", "category"]).columns

for col in numerical_cols:
    if data[col].isnull().sum() > 0:
        data[col].fillna(data[col].mean(), inplace=True)

for col in categorical_cols:
    if data[col].isnull().sum() > 0:
        data[col].fillna(data[col].mode()[0], inplace=True)

# 3. Identify Duplicates
print("\n--- Duplicates ---")
duplicate_rows = data.duplicated().sum()
print(f"Number of duplicate rows: {duplicate_rows}")

if duplicate_rows > 0:
    data = data.drop_duplicates()
    print(f"Data shape after removing duplicates: {data.shape}")

# 4. Summary Statistics
print("\n--- Summary Statistics ---")
print(data.describe())

# 5. Detect Outliers using Z-Score
print("\n--- Detecting Outliers with Z-Score ---")
z_scores = np.abs(stats.zscore(data.select_dtypes(include=[np.number])))
threshold = 3
outliers = (z_scores > threshold).any(axis=1)
print(f"Number of outliers: {outliers.sum()}")

# Optionally remove outliers
data = data[~outliers]
print(f"Data shape after removing outliers: {data.shape}")

# 6. Validate Categorical Data
print("\n--- Categorical Data Validation ---")
for col in categorical_cols:
    unique_vals = data[col].unique()
    print(f"\nUnique values in '{col}':\n", unique_vals)
    # Example: Standardize to lowercase
    data[col] = data[col].str.lower()
    # Example: Replace known inconsistencies
    # data[col] = data[col].replace({'old_value': 'new_value'})

# 7. Ensure Data Consistency and Integrity
print("\n--- Data Consistency Checks ---")
# Example: Date consistency
if "manufacture_date" in data.columns and "expiry_date" in data.columns:
    data["manufacture_date"] = pd.to_datetime(data["manufacture_date"], errors="coerce")
    data["expiry_date"] = pd.to_datetime(data["expiry_date"], errors="coerce")
    invalid_dates = data[data["expiry_date"] < data["manufacture_date"]]
    print(f"Records with expiry_date before manufacture_date: {invalid_dates.shape[0]}")
    # Remove invalid dates
    data = data[data["expiry_date"] >= data["manufacture_date"]]

# Example: Logical numerical relationships
if "calories" in data.columns:
    negative_calories = data[data["calories"] < 0].shape[0]
    print(f"Records with negative calories: {negative_calories}")
    # Remove negative calories
    data = data[data["calories"] >= 0]

print("\n--- Final Data Shape ---")
print(data.shape)
\end{lstlisting}

% Continue with other sections and listings...

The code in Listing~\ref{lst:validate_data} performs the following tasks:
  \begin{itemize}
      \item Load the data from the CSV file
      \item Calculate the percentage of missing values per column
      \item Identify columns with more than 90\% missing values
      \item Drop columns with more than 90\% missing values
      \item Save the cleaned data to a new CSV file
  \end{itemize}
The cleaned data is saved to a new CSV file called \texttt{branded\_food\_cleaned.csv}. The next step is to analyze the data. The resons for dropping columns with more than 90\% missing values is that they are not useful for this type of anlaysis and add unnecessary noise to the data.

\section{Re-evaluate Data source}

I found the data source at \url{https://fdc.nal.usda.gov/} to be insufficient for my analysis. The data is missing key information such as the source of the data. I will look for a new data source. After a bit of research, I found a new data source at \url{https://www.nutritionix.com/database} that has the data I need. I will use this data source for the rest of the analysis. I will download the data and inspect it as I did with the previous data source. I noticed that there are quotation marks in the data. I will remove them.

\begin{lstlisting}[style=bashStyle, caption={Remove Quotation Marks}, label={lst:remove_quotation}]
tr -d "\"'" < openfoodfacts.csv > cleaned_openfoodfacts.csv
\end{lstlisting}
As shown in Listing~\ref{lst:remove_quotation}, the command removes the quotation marks from the file \texttt{openfoodfacts.csv} and saves the cleaned data to \texttt{cleaned\_openfoodfacts.csv}. I will use the cleaned data for the rest of the analysis. I then trimed the data to 100,000 rows just to see what we are working with see Listing \texit{trim.py} 


\end{document}

