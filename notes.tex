
\documentclass{report} % Changed from 'article' to 'report'

\title{Notes On Data}
\author{Derek R Neilson}
\date{\today}

\usepackage{geometry}    % Adjust margins
\usepackage{graphicx}    % For including images
\usepackage{amsmath}     % For mathematical expressions
\usepackage{hyperref}    % For hyperlinks
\usepackage{booktabs}    % For better tables
\usepackage{url}         % For URLs
\usepackage{listings}    % For code listings
\usepackage{xcolor}      % For custom colors (optional)
\usepackage{hyphenat}    % For hyphenation

% Adjust margins
\geometry{
    a4paper,
    margin=1in
}

% Base style for all listings
\lstdefinestyle{baseStyle}{
    basicstyle=\ttfamily\small,          % Use a smaller monospace font
    numbers=left,                        % Line numbers on the left
    numberstyle=\tiny\color{gray},       % Style for line numbers
    stepnumber=1,                        % Number every line
    numbersep=5pt,                       % Space between numbers and code
    backgroundcolor=\color{white},       % Background color
    frame=single,                        % Single frame around code
    breaklines=true,                     % Allow line breaking
    breakatwhitespace=false,            % Allow breaks at any character
    breakautoindent=true,
    breakindent=10pt,                    % Indent for broken lines
    columns=flexible,                    % Better alignment
    keepspaces=true,                     % Preserve spaces
    showstringspaces=false,              % Don't show spaces in strings
    tabsize=4,                           % Set tab size
    captionpos=b,                        % Caption below the listing
    postbreak=\mbox{\textcolor{red}{$\hookrightarrow$}\space}, % Symbol indicating a line break
    literate={/}{/}1{.}{.}1{-}{-}1{_}{\_}1,  % Allow breaks after / . - _
}

% Style for Python
\lstdefinestyle{pythonStyle}{
    style=baseStyle,                     % Inherit from baseStyle
    language=Python,                     % Set language
    keywordstyle=\color{blue}\bfseries,  % Keywords in blue bold
    commentstyle=\color{gray}\itshape,   % Comments in gray italic
    stringstyle=\color{orange},          % Strings in orange
    morekeywords={self, cls, async, await}, % Additional Python keywords
}

% Style for R
\lstdefinestyle{rStyle}{
    style=baseStyle,
    language=R,
    keywordstyle=\color{purple}\bfseries,
    commentstyle=\color{gray}\itshape,
    stringstyle=\color{red},
    morekeywords={<-},                   % Assignment operator in R
}

% Style for Bash
\lstdefinestyle{bashStyle}{
    style=baseStyle,
    language=bash,
    keywordstyle=\color{blue}\bfseries,
    commentstyle=\color{gray}\itshape,
    stringstyle=\color{orange},
    morekeywords={wget, unzip, rm, sftp, ssh, put, bye}, % Additional Bash keywords
}

% Style for SQL (as an example of another data science-related language)
\lstdefinestyle{sqlStyle}{
    style=baseStyle,
    language=SQL,
    keywordstyle=\color{blue}\bfseries,
    commentstyle=\color{gray}\itshape,
    stringstyle=\color{orange},
    morekeywords={SELECT, FROM, WHERE, INSERT, UPDATE, DELETE, JOIN, INNER, OUTER},
}

% Style for Julia
\lstdefinestyle{juliaStyle}{
    style=baseStyle,
    language=Julia,
    keywordstyle=\color{teal}\bfseries,
    commentstyle=\color{gray}\itshape,
    stringstyle=\color{magenta},
    morekeywords={end, function, using, import},
}

% Style for MATLAB
\lstdefinestyle{matlabStyle}{
    style=baseStyle,
    language=Matlab,
    keywordstyle=\color{blue}\bfseries,
    commentstyle=\color{gray}\itshape,
    stringstyle=\color{orange},
    morekeywords={function, end, if, else, for, while},
}

% Style for LaTeX code
\lstdefinestyle{latexStyle}{
    style=baseStyle,
    language=TeX,
    keywordstyle=\color{blue}\bfseries,
    commentstyle=\color{gray}\itshape,
    stringstyle=\color{orange},
    morekeywords={\documentclass, \usepackage, \begin, \end, \section, \subsection},
}


\begin{document}

\maketitle

\begin{abstract}
  This document contains notes on the data. The notes are intended to demonstrate how I filter and manipulate the data. The primary focus is on cleaning the data, handling missing values, and transforming data to suit the analytical objectives. These notes serve as a comprehensive guide for understanding my data processing workflow. 
\end{abstract}

\documentclass{article}

\title{Notes On Data}
\author{Derek R Neilson}
\date{\today}

\usepackage{graphicx} % For including images
\usepackage{amsmath}  % For mathematical expressions
\usepackage{hyperref} % For hyperlinks
\usepackage{booktabs} % For better tables
\usepackage{url}      % For URLs
\usepackage{listings}   % For code listings
\usepackage{xcolor}     % For custom colors (optional)

\lstset{
    basicstyle=\ttfamily\small,
    keywordstyle=\color{blue},
    commentstyle=\color{gray},
    stringstyle=\color{orange},
    numbers=left,
    numberstyle=\tiny\color{gray},
    stepnumber=1,
    numbersep=5pt,
    backgroundcolor=\color{white},
    frame=single,
    breaklines=true,
    captionpos=b,        % Position of the caption (b=below, t=top)
    language=bash        % Specify the language for syntax highlighting

\begin{document}

\maketitle

\begin{abstract}
  This document contains notes on the data. The notes are intended to demonstrate how I filter and manipulate the data, and are purely for my instructor to review.
\end{abstract}

\section{Introduction}

In data analysis, the ability to effectively filter and manipulate data is crucial for extracting meaningful insights. This document outlines the methodologies and tools I employ to pre-process and analyze the dataset. The primary focus is on cleaning the data, handling missing values, and transforming data to suit the analytical objectives. These notes serve as a comprehensive guide for understanding my data processing workflow.

\section{Data Collection}
The data was collected from \url{https://fdc.nal.usda.gov/}. The dataset is 2.9GB and is labeled \texttt{Branded} and is in JSON format. I chose this dataset because it is large and one can assume that it has the most rows because it is so large. 

To download the data, I used the following commands: 


\
baegin{lstlisting}[caption={Download, Extract, and Remove Dataset File}, label={lst:download_extract}]
wget https://fdc.nal.usda.gov/fdc-datasets/FoodData_Central_branded_food_json_2024-04-18.zip
unzip FoodData_Central_branded_food_json_2024-04-18.zip 
rm FoodData_Central_branded_food_json_2024-04-18.zip
\end{lstlisting}

As shown in Listing~\ref{lst:download_extract}, the commands download, extract, and remove the dataset file.

It is worth noteing that I am using git to track changes in the code and data. The git commands will not be shown in this documen for brevity sake.

\section{Data Inspection}
I receved the following files after extracting:
  \begin{itemize}
    \item \texttt{brandedDownload.json} I am assuming that this is the main file
    \item \texttt{foundationDownload.json} I am assuming that this is a supporting file
  \end{itemize}


\end{document}



\end{document}

